\documentclass{scrreprt}
\usepackage{listings}
\usepackage{underscore}
\usepackage{graphicx}
\usepackage[bookmarks=true]{hyperref}
\usepackage[utf8]{inputenc}
\usepackage[german]{babel}
\hypersetup{
    bookmarks=false,    % show bookmarks bar?
    pdftitle={Storycard Template},    % title
    pdfauthor={Tobias Fischer, Miklós Libak, Eva Norkunas, Lucas Weinknecht}, % author
    pdfsubject={TeX and LaTeX},                        % subject of the document
    pdfkeywords={TeX, LaTeX, graphics, images}, % list of keywords
    colorlinks=true,       % false: boxed links; true: colored links
    linkcolor=blue,       % color of internal links
    citecolor=black,       % color of links to bibliography
    filecolor=black,        % color of file links
    urlcolor=purple,        % color of external links
    linktoc=page            % only page is linked
}%
\def\myversion{0.1 }
\date{}
%\title
\usepackage{hyperref}

\newcommand{\lstinlinejava}[1]{\lstinline[language=java]{#1}}
\newcommand{\lstj}[1]{\lstinlinejava{#1}}

\usepackage{color}

\definecolor{pblue}{rgb}{0.13,0.13,0.7}
\definecolor{pgreen}{rgb}{0,0.5,0}
\definecolor{pred}{rgb}{0.9,0,0}
\definecolor{pgrey}{rgb}{0.46,0.45,0.48}

\lstset{language=Java,
  showspaces=false,
  showtabs=false,
  breaklines=true,
  showstringspaces=false,
  breakatwhitespace=true,
  commentstyle=\color{pgreen},
  keywordstyle=\color{pblue},
  stringstyle=\color{pred},
  basicstyle=\ttfamily,
  moredelim=[il][\textcolor{pgrey}]{$$},
  moredelim=[is][\textcolor{pgrey}]{\%\%}{\%\%}
}

\begin{document}

\begin{flushright}
    \rule{16cm}{5pt}\vskip1cm
    \begin{bfseries}
        \Huge{STORYCARD TEMPLATE EXPLORATION}\\
        \vspace{1.5cm}
        for\\
        \vspace{1.5cm}
        MELT Chess\\
        \vspace{1.5cm}
        \LARGE{Version \myversion}\\
        \vspace{1.5cm}
        \vspace{1.5cm}
        \today\\
    \end{bfseries}
\end{flushright}

\newpage
\section{Storycard Issues}
\subsection*{\#2 Erstellen der Game Klasse}
Milestone: 1. Iteration\\

\begin{itemize}
\item[Priorisierung] B
\item[Storypoints] 3
\item[Risiko] 0
\end{itemize}

Die ``Game`` Klasse soll den Gesamtzustand einer Partie kapseln, aber hauptsächlich für die beiden GUI Module relevant sein.

\textbf{Abgeschlossen wenn}
\begin{itemize}
\item $[x]$ Funktion für "neues Spiel starten"
\item $[ ]$ Verwaltet alle nötigen Objekte einer Partie und bietet ein einfaches Interface für die GUI Module
\item $[ ]$ Erkennt Schachmatt und Patt (\#13 )
\item $[x]$ ``Board`` Objekte in einer Liste als Gamehistory merken (BONUS)
\item $[ ]$ Funktionen zum zurück oder vorwärs gehen in der Gamehistory
\item $[ ]$ Verwaltung der Schachuhren?
\item $[ ]$ Sichern und Laden des Spielzustands
\end{itemize}

\end{document}
